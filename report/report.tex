% https://www.springer.com/gp/computer-science/lncs/conference-proceedings-guidelines
\documentclass[runningheads]{llncs}
%
\usepackage[T1]{fontenc}
% T1 fonts will be used to generate the final print and online PDFs,
% so please use T1 fonts in your manuscript whenever possible.
% Other font encondings may result in incorrect characters.
%
\usepackage{graphicx}
\usepackage{float}
\usepackage{hyperref}
\usepackage{amsmath}
\usepackage{parselines}
\usepackage{listings}
% Used for displaying a sample figure. If possible, figure files should
% be included in EPS format.
%
% If you use the hyperref package, please uncomment the following two lines
% to display URLs in blue roman font according to Springer's eBook style:
%\usepackage{color}
%\renewcommand\UrlFont{\color{blue}\rmfamily}
%\urlstyle{rm}
%
\begin{document}
%
% todo: set this
\title{CRC - Incipient Cognition}


%
%\titlerunning{Abbreviated paper title}
% If the paper title is too long for the running head, you can set
% an abbreviated paper title here
%
\author{Miguel Albuquerque\inst{1}\orcidID{1105828} \and
Pedro Baptista\inst{1}\orcidID{96302} \and colega3\inst{1}\orcidID{n3}}

\authorrunning{Group 28?}
\institute{Instituto Superior Técnico, Av. Rovisco Pais 1, 1049-001 Lisboa, Portugal \and
\email{\{miguel.albuquerque,pedro.maria.baptista, colega3}@tecnico.ulisboa.pt}

%
\maketitle              % typeset the header of the contribution
%
\begin{abstract}
OUR ABSTRACT HERE
\end{abstract}


\keywords{Dkron \and Distributed \and Benchmark \and Scalability \and Consensus \and Dkron Server \ and Dkron Agent \and Raft.}
%
%
%
% %%%
% %%%


\section{Introduction}

Relation to chosen paper
\begin{enumerate}
    \item Background
    \item Motivation
    \item Rel. Work
    \item Easy to deploy with built-in replicated storage (BuntDB) relying on the Raft protocol.
    \item Provides a Web-GUI for administration.
\end{enumerate}

% FIGURE EXAMPLE
%\begin{figure}[h]
%\centering
%%\caption{dkron Logo .} \label{fig1}
%\end{figure}

\section{Method/Model}
What did they do -> baseline
What we did; what we added/changed


\section{Results + Discussion}

Dis we get the same insights? Assert (our results, baseline)
What else did we find?

\section{Conclusion}

\subsection{Limitations}

\subsection{Future Work}

\section{Annexes}



\subsection{Metrics Visualisation}

% more figure examples
%\begin{figure}[H]
%    \centering
%    \includegraphics[width=1\textwidth]{media/FailureRate_trend.png} \caption{Failure Rate Trend Across Benchmarks}
%    \label{fig:aaa}
%\end{figure}



% ---- Bibliography ----
%
% BibTeX users should specify bibliography style 'splncs04'.
% References will then be sorted and formatted in the correct style.
%
% \bibliographystyle{splncs04}
% \bibliography{mybibliography}
%
\begin{thebibliography}{8}

\bibitem{mm2024}
Matos, M.: IST - <Course name here> - Class slides
Lisbon (2024)

\bibitem{dkron}
https://dkron.io/docs/intro
\bibitem{dkron2}
https://k6.io/docs/
\bibitem{dkron3}
https://helm.sh/docs/
\bibitem{dkron4}
https://kubernetes.io/docs/home/
\end{thebibliography}
\end{document}
